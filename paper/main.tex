\documentclass[11pt]{article}
\usepackage{amsmath, amssymb, amscd, amsthm, amsfonts}
\usepackage{graphicx}

\oddsidemargin 0pt
\evensidemargin 0pt
\marginparwidth 40pt
\marginparsep 10pt
\topmargin -20pt
\headsep 10pt
\textheight 8.7in
\textwidth 6.65in
\linespread{1.2}

\title{Simulating HPC Application Patterns with Spatter and SST}
\author{Rex Bianchi}
\date{August 1, 2025}


\begin{document}

\maketitle

\begin{abstract}
%This is a working abstract from project description, REWRITE THIS AT SOME POINT
Spatter is a memory benchmark created for the purpose of studying irregular memory access patterns on CPUs and GPUs. By collecting traces of complex scientific applications and translating them into Spatter inputs, users can mimic their memory access behavior which greatly simplifies the process of evaluating memory systems on new processors beyond simple metrics such as latency and bandwidth. It has been selected by several Department of Energy (DoE) labs to be used for next-generation system procurements.

This goal of this project is to extend the capabilities of Spatter and to utilize it to evaluate several key applications of interest to DoE lab partners. This specific project will focus on co-design using the Structural Simulation Toolkit (SST) and the new Spatter-SST extension. 
\end{abstract}

\section{Introduction}

Introduction goes here, I'll be writing content here. Awesome content. More Awesome content

\section{Related work}

Related work goes here. This is more, more work. Test citation goes here \cite{sheridan:2024:workflow_memsys}

\bibliographystyle{alpha}
\bibliography{references} % see references.bib for bibliography management

\end{document}